\subsection{Top-up Injection}

After the commissioning phase, the Sirius injection system will operate in top-up mode and a control application will be necessary to conduct the required periodic injections. A simplified version of this HLA was developed to test the simulation of the injection process in VACA. It consists of a PyQt application that monitors the beam current in the storage ring and starts the injection cycle in order to mantain the current level within the desired tolerance. The connection between the CA protocol and the Python language was done with the PyEpics package~\cite{pyepics}. The user interface contains a display of the beam current over time and the fill pattern in the storage ring, as shown in Fig. \ref{fig:topup}.

In order to include this application in the Sirius control system, several functionalities must be added. For example, in this version, it is only possible to 
choose between multi-bunch or single-bunch injection mode. However, as the timing system will support the injection to any bucket, the final version of this HLA should allow the specification of any filling pattern.


\begin{figure}[!htb]
%   \vspace*{-.5\baselineskip}
   \centering
   \includegraphics*[width=150pt]{WEPOPRPO22f4}
   \caption{Top-up injection display application developed with the PyQt framework.}
   \label{fig:topup}
%   \vspace*{-\baselineskip}
\end{figure}

